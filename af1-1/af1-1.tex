\documentclass[11pt, a4paper]{article}
\usepackage[utf8]{inputenc}
\usepackage[portuguese]{babel}
\usepackage[T1]{fontenc}
\usepackage[margin=2cm]{geometry}
\usepackage{amsmath}
\usepackage{amsfonts}
\usepackage{amsthm}
\usepackage{enumitem}
\usepackage{csquotes}
\usepackage[citestyle=authoryear]{biblatex}
\usepackage[hidelinks]{hyperref}

\addbibresource{./bibliografia.bib}

\title{
	Lógica e Teoria de Conjuntos\\
	Actividade Formativa 1.1\\
	Proposta de Resolução
}

\author{
	Carlos Pinto Machado
	<\href{mailto:2200909@estudante.uab.pt}{2200909@estudante.uab.pt}>
}

\date{\today}

\hypersetup{
	pdfsubject = {Lógica e Teoria de Conjuntos},
	pdftitle = {Actividade Formativa 1.1(Proposta de Resolução)},
	pdfauthor = {Carlos Pinto Machado(2200909)},
	pdfcreator = {},
	pdfproducer = {}
}

\makeindex

\newcommand{\exercicio}[1]{
	\section*{Exercício #1}
	\phantomsection
	\addcontentsline{toc}{section}{Exercício #1}
}

\newcommand{\alinea}[1]{
	\subsection*{(#1)}
	\phantomsection
	\addcontentsline{toc}{subsection}{(#1)}
}


\begin{document}
\maketitle

\section*{Introdução}
\addcontentsline{toc}{section}{Introdução}

\paragraph{} Resolução dos exercícios de \cite[secção 1.1 Linguagem]{Edmundo2021}.


\exercicio{1.1.3}

\begin{enumerate}[label=(\alph*)]
	\item
	\item
	\item
	\item
	\item
	\item
	\item
	\item
	\item
	\item
	\item
\end{enumerate}

\exercicio{1.1.8}

\begin{enumerate}[label=(\alph*)]
	\item
	\item
	\item
	\item
	\item
	\item
\end{enumerate}

\exercicio{1.1.11}

\begin{enumerate}[label=(\alph*)]
	\item
	\item
	\item
	\item
	\item
\end{enumerate}

\exercicio{1.1.20}

\begin{enumerate}[label=(\alph*)]
	\item\hfill
		\begin{enumerate}[label=(\roman*)]
			\item 
			\item 
			\item 
			\item 
		\end{enumerate}
	\item\hfill
		\begin{enumerate}[label=(\roman*)]
			\item 
			\item 
			\item 
			\item 
		\end{enumerate}
	\item\hfill
		\begin{enumerate}[label=(\roman*)]
			\item 
			\item 
			\item 
			\item 
		\end{enumerate}
\end{enumerate}

\exercicio{1.1.23}

\begin{enumerate}[label=(\alph*)]
	\item
	\item
	\item
\end{enumerate}


\clearpage

\printbibliography[title={Bibliografia},heading=bibintoc]

\end{document}
